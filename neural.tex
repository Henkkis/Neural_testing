\documentclass[border=1mm]{standalone}
\usepackage{xstring}
\usepackage{xparse}
\usepackage{tikz}
\usetikzlibrary{calc}

\ExplSyntaxOn
\NewDocumentCommand{\setarray}{O{;}mm}
 {
  \seq_clear_new:c { g_alexb_array_#2_seq }
  \seq_gset_split:cnn { g_alexb_array_#2_seq } { #1 } { #3 }
 }
\NewDocumentCommand{\getfromarray}{mm}
 {
  \seq_item:cn { g_alexb_array_#1_seq } { #2 }
 }
\cs_generate_variant:Nn \seq_gset_split:Nnn { c }
\cs_set_eq:NN \inteval \int_eval:n
\ExplSyntaxOff

\begin{document}
\pagestyle{empty}




\def\layersep{2.5cm}
\input{neural_params.tex}
%\def\innum{5}
%\def\outnum{5}
%\def\numhidden{3}
%\def\networkstruct{3,4,5}
%\setarray{Nco}{3;4;5;5}


\newcounter{distmul}
\newcounter{temp}


\begin{tikzpicture}[shorten >=1pt,->,draw=black!50, node distance=\layersep]
    \tikzstyle{every pin edge}=[<-,shorten <=1pt]
    \tikzstyle{neuron}=[circle,fill=black!25,minimum size=17pt,inner sep=0pt]
    \tikzstyle{input neuron}=[neuron, fill=green!50];
    \tikzstyle{output neuron}=[neuron, fill=red!50];
    \tikzstyle{hidden neuron}=[neuron, fill=blue!50];
    \tikzstyle{annot} = [text width=4em, text centered]

    % Draw the input layer nodes
    \foreach \name / \y in {1,...,\innum}
   		 \path[yshift=\innum*0.5cm]
        node[input neuron, pin=left:Input \#\y] (H0-\name) at (0,-\y) {};

    % Draw the hidden layer nodes
    \foreach \id / \x in \networkstruct
    \stepcounter{distmul}
    \foreach \name / \y in {1,...,\x}
        \path[yshift=\x*0.5cm]
            node[hidden neuron] (H\thedistmul-\name) at (\thedistmul*\layersep,-\y cm) {};
    	;
        
        
  %  \foreach \name / \y in {1,...,5}
  %      \path[yshift=1cm]
%	    node[hidden neuron] (H2-\name) at (2*\layersep,-\y cm) {};

    % Draw the output layer node
    \stepcounter{distmul}
      \foreach \name / \y in {1,...,\outnum}
      \path[yshift=\outnum*0.5cm]
    node[output neuron,pin={[pin edge={->}]right:Output \#\y}] (H\thedistmul-\name) at (\thedistmul*\layersep,-\y cm) {};

    % Connect every node in the input layer with every node in the
    % hidden layer.


   \foreach \source in {1,...,\innum}
 	\foreach \dest in {1,...,\getfromarray{Nco}{1}}
            \path (H0-\source) edge (H1-\dest);
           ;
           
 \stepcounter{temp}
  \setcounter{distmul}{0}         
      \foreach \name / \x in \networkstruct
          \stepcounter{temp}
          \stepcounter{distmul}
           \foreach \source in {1,...,\x}
           		\foreach \dest in {1,...,\getfromarray{Nco}{\thetemp}}
                \path (H\thedistmul-\source) edge (H\thetemp-\dest);
                    

    % Connect every node in the hidden layer with the output layer
   % \foreach \source in {1,...,5}
   %     \path (H-\source) edge (O);

    % Annotate the layers
    % \foreach \id / \x in \networkstruct
    %  \node[annot,above of=H\id-1, node distance=1cm] (hl\id) {Hidden layer};
   % \node[annot,left of=hl] {Input layer};
   % \node[annot,right of=hl] {Output layer};
\end{tikzpicture}
% End of code
\end{document}
